\documentclass[12pt]{article}

\usepackage{amsfonts, amsmath, amssymb, amstext, latexsym}
\usepackage{graphicx, epsfig}
\usepackage[latin1]{inputenc}
\usepackage[french]{babel}
\usepackage{exscale}
\usepackage{amsbsy}
\usepackage{amsopn}
\usepackage{fancyhdr}
\usepackage{listings}

\newcommand{\noi}{\noindent}
\newcommand{\dsp}{\displaystyle}
\newcommand{\ind}{{{\large 1} \hspace*{-1.6mm} {\large 1}}}


\textheight 25cm
\textwidth 16cm
\oddsidemargin 0cm
\evensidemargin 0cm
\topmargin 0cm
\hoffset -0mm
\voffset -20mm

\title{TP OpenStreetMap - SIG}
\author{CARRE Ludovic \and DELOCHE Maxime \and LEFOULON Vincent}

\begin{document}

\maketitle

\section*{Question 1}

\begin{lstlisting}[language=SQL]
SELECT COUNT(*)
FROM users;
\end{lstlisting}

\begin{lstlisting}
count 
-------
  4576
  (1 row)
\end{lstlisting}

\newpage

\section*{Question 2.a}

\begin{lstlisting}[language=SQL]
SELECT ST_X(geom), ST_Y(geom), ST_Z(geom)
FROM nodes
WHERE id = 1787038609;
\end{lstlisting}

\begin{lstlisting}
st_x      |   st_y    | st_z 
----------+-----------+------
5.7680106 | 45.192893 |     
(1 row)
\end{lstlisting}

\section*{Question 2.b}

\begin{lstlisting}[language=SQL]
SELECT *
FROM spatial_ref_sys
WHERE srid = (
    SELECT ST_SRID(geom) FROM nodes WHERE id = 1787038609
);
\end{lstlisting}

Le système de coordonnées utilisé est WGS84, c'est-à-dire le système GPS.

\newpage

\section*{Question 3}

\begin{lstlisting}[language=SQL]
SELECT ST_AsEWKT(ST_Centroid(linestring))
FROM ways
WHERE tags->'amenity' = 'townhall' AND tags->'name' like '%Grenoble%';
\end{lstlisting}

\begin{lstlisting}
st_asewkt                      
----------------------------------------------------
 SRID=4326;POINT(5.73643908557793 45.1864548121024)
 (1 row)
\end{lstlisting}

\newpage

\section*{Question 4}

\begin{lstlisting}[language=SQL]
SELECT tags->'highway', COUNT(id)
FROM ways
WHERE tags ? 'highway'
GROUP BY tags->'highway'
ORDER BY COUNT(id) DESC;
\end{lstlisting}

\begin{lstlisting}
?column?                 | count 
-------------------------+-------
residential              | 94356
unclassified             | 77678
service                  | 64461
track                    | 58276
tertiary                 | 21923
footway                  | 21742
path                     | 21413
secondary                | 18444
...
\end{lstlisting}

\newpage

\section*{Question 5.a}

\begin{lstlisting}[language=SQL]
SELECT tags->'highway', SUM(ST_Length(linestring))
FROM ways
WHERE tags ? 'highway'
GROUP BY tags->'highway'
ORDER BY SUM(ST_Length(linestring)) DESC;
\end{lstlisting}

\begin{lstlisting}
?column?                 |         sum          
-------------------------+----------------------
unclassified             |     435.285712879273
track                    |     343.366696791793
residential              |     210.396982404606
tertiary                 |     208.294200085326
path                     |     134.367018999509
...
\end{lstlisting}

\section*{Question 5.b}

La documentation dit : \textit{Geometry: Measurements are in the units of the spatial reference system of the geometry.}

\begin{lstlisting}[language=SQL]
SELECT *
FROM spatial_ref_sys
WHERE srid = (
    SELECT ST_SRID(linestring) FROM ways LIMIT 1
);
\end{lstlisting}

Le système de référence est WGS84 et l'unité est le degré (\textit{degree}).

\section*{Question 5.c}

On travaille avec des géographies, dont l'unité est le mètre.

\begin{lstlisting}[language=SQL]
SELECT tags->'highway', SUM(ST_Length(linestring::geography))/1000
FROM ways
WHERE tags ? 'highway'
GROUP BY tags->'highway'
ORDER BY SUM(ST_Length(linestring)) DESC;
\end{lstlisting}

\newpage

\section*{Question 6}

On récupère tous les bâtiments de l'Ensimag depuis la table \verb?relation_members?.

\begin{lstlisting}[language=SQL]
SELECT SUM(ST_Area(ST_MakePolygon(ways.linestring)::geography))
FROM relations
JOIN relation_members ON relations.id = relation_members.relation_id
JOIN ways ON relation_members.member_id = ways.id
WHERE
    relations.tags->'name' = 'Ensimag' AND
    relation_members.member_type = 'W';
\end{lstlisting}

\begin{lstlisting}
       sum
------------------
 2640.86233445186
(1 ligne)
\end{lstlisting}

\newpage

\section*{Question 7}

On considère qu'une école est contenue dans un quartier si sa zone géographique (\textit{linestring})
est incluse dans celle du quartier. On commence par déterminer le système
géographique utilisé par la table \verb?quartier? :

\begin{lstlisting}[language=SQL]
SELECT ST_SRID(the_geom)
FROM quartier
LIMIT 1;
\end{lstlisting}

\begin{lstlisting}[language=SQL]
st_srid
---------
 2154
(1 ligne)
\end{lstlisting}

Puis on convertit les écoles en polygônes et les transforme dans le bon
système géographique avant de vérifier la relation d'inclusion avec les quartiers :

\begin{lstlisting}[language=SQL]
SELECT quartier.quartier, count(ways.id)
FROM quartier, ways
WHERE
    ways.tags->'amenity' = 'school' AND
    ST_Contains(
        quartier.the_geom,
        ST_Transform(
            ST_MakePolygon(
                ST_AddPoint(
                    ways.linestring,
                    ST_StartPoint(ways.linestring)
                )
            ),
            2154
        )
    )
GROUP BY quartier.quartier
ORDER BY count(ways.id) DESC;
\end{lstlisting}

\begin{lstlisting}
quartier             | count 
---------------------+-------
BERRIAT ST BRUNO     |    13
CENTRE VILLLE        |    12
EXPOSITION-BAJATIERE |     9
RONDEAU-LIBERATION   |     7
EAUX CLAIRES         |     6
ABBAYE-JOUHAUX       |     5
...
\end{lstlisting}

\newpage

\section*{Question 8 (incomplet)}

On définit le centre géographique d'une région comme la municipalité ayant une
distance minimale avec sa municipalité la plus éloignée. On définit la distance
entre deux villes comme celle entre les centroïdes des centroïdes de leurs bâtiments.

On commence par récupérer les municipalités :

\begin{lstlisting}[language=SQL]
SELECT *
FROM relations
WHERE tags->'boundary' = 'administrative' AND tags->'admin_level' = '8';
\end{lstlisting}

Une relation contient plusieurs éléments géométriques, listés dans la table
\verb?relation_members? :

\begin{lstlisting}[language=SQL]
SELECT *
FROM relations
INNER JOIN relation_members ON relations.id = relation_members.relation_id
WHERE tags->'boundary' = 'administrative' AND tags->'admin_level' = '8';
\end{lstlisting}

\verb?relation_members.member_type? fait référence à la table contenant les données
géographiques :

\begin{lstlisting}[language=SQL]
SELECT DISTINCT(member_type)
FROM relation_members;
\end{lstlisting}

\begin{lstlisting}
member_type 
-------------
W
R
N
(3 rows)
\end{lstlisting}

Les bâtiments d'une ville sont stockés dans la table \verb?ways?. On calcule le centroïd
de chaque municipalité :

\begin{lstlisting}[language=SQL]
SELECT relations.id, ST_Centroid(ST_Union(ways.linestring))
FROM relations
JOIN relation_members ON relations.id = relation_members.relation_id
JOIN ways ON relation_members.member_id = ways.id
WHERE
    relations.tags->'boundary' = 'administrative' AND
    relations.tags->'admin_level' = '8' AND
    relation_members.member_type = 'W'
GROUP BY relations.id;
\end{lstlisting}

TODO : ce que contient la table ways pour les municipalités est étrange. Notamment,
il n'y a aucune ligne avec building=yes.

Puis, pour chaque ville, on calcule sa distance aux autres, prend celle maximale
et retourne la ville ayant la plus petite :

\begin{lstlisting}[language=SQL]
\end{lstlisting}

\newpage

\section*{Question 9 (incomplet)}

Par "endroit", on comprendra "point géographique". La requête renverra, s'il
existe, un \verb?Point? PostGIS éloigné d'au moins 10 kilomètres de tout bâtiment.

Commençons par récupérer les bâtiments :

\begin{lstlisting}[language=SQL]
SELECT *
FROM ways
WHERE tags->'building' = 'yes';
\end{lstlisting}

\newpage

\section*{Documentation de l'implémentation Java}

une documentation concernant votre application Java, contenant entre autres un descriptif rapide des fonctionnalités, de l'architecture, des requêtes effectuées, des choix de conception, et des tests effectués (et leur résultat). Cette documentation doit être concise et précise (inutile de faire 42 pages de documentation ; vous n'aurez pas le temps de toute façon).

\subsection*{Usage}

La commande \verb?make? permet de compiler le projet. \verb?make run? l'exécute.

TODO : fonctionnalités implémentées

\subsection*{Architecture du code}

Le code se base sur une architecture basique MVC :

\begin{itemize}
\item Il n'y a qu'un seul contrôleur, le fichier \verb?Main.java?
\item Le package \verb?models? contient les classes effectuant les requêtes à la
base de données
\item Les classes du package \verb?views? se chargent de l'affichage graphique
\end{itemize}

TODO : choix de conception particuliers ?

\subsection*{Requêtes SQL}



\subsection*{Tests}

\end{document}
