\documentclass[12pt]{article}

\usepackage{amsfonts, amsmath, amssymb, amstext, latexsym}
\usepackage{graphicx, epsfig}
\usepackage[latin1]{inputenc}
\usepackage[french]{babel}
\usepackage{exscale}
\usepackage{amsbsy}
\usepackage{amsopn}
\usepackage{fancyhdr}
\usepackage{listings}

\newcommand{\noi}{\noindent}
\newcommand{\dsp}{\displaystyle}
\newcommand{\ind}{{{\large 1} \hspace*{-1.6mm} {\large 1}}}


\textheight 25cm
\textwidth 16cm
\oddsidemargin 0cm
\evensidemargin 0cm
\topmargin 0cm
\hoffset -0mm
\voffset -20mm

\title{TP OpenStreetMap - SIG}
\author{CARRE Ludovic \and DELOCHE Maxime \and LEFOULON Vincent}

\begin{document}

\maketitle

\section*{Question 1}

\begin{lstlisting}[language=SQL]
SELECT COUNT(*)
FROM users;
\end{lstlisting}

\begin{lstlisting}
count 
-------
  4576
  (1 row)
\end{lstlisting}

\newpage

\section*{Question 2.a}

\begin{lstlisting}[language=SQL]
SELECT ST_X(geom), ST_Y(geom), ST_Z(geom)
FROM nodes
WHERE id = 1787038609;
\end{lstlisting}

\begin{lstlisting}
st_x      |   st_y    | st_z 
----------+-----------+------
5.7680106 | 45.192893 |     
(1 row)
\end{lstlisting}

\section*{Question 2.b}

\begin{lstlisting}[language=SQL]
SELECT *
FROM spatial_ref_sys
WHERE srid = (
    SELECT ST_SRID(geom) FROM nodes WHERE id = 1787038609
);
\end{lstlisting}

Le syst�me de coordonn�es utilis� est WGS84, c'est-�-dire le syst�me GPS.

\newpage

\section*{Question 3}

\begin{lstlisting}[language=SQL]
SELECT ST_AsEWKT(ST_Centroid(linestring))
FROM ways
WHERE tags->'amenity' = 'townhall' AND tags->'name' like '%Grenoble%';
\end{lstlisting}

\begin{lstlisting}
st_asewkt                      
----------------------------------------------------
 SRID=4326;POINT(5.73643908557793 45.1864548121024)
 (1 row)
\end{lstlisting}

\newpage

\section*{Question 4}

\begin{lstlisting}[language=SQL]
SELECT tags->'highway', COUNT(id)
FROM ways
WHERE tags ? 'highway'
GROUP BY tags->'highway'
ORDER BY COUNT(id) DESC;
\end{lstlisting}

\begin{lstlisting}
?column?                 | count 
-------------------------+-------
residential              | 94356
unclassified             | 77678
service                  | 64461
track                    | 58276
tertiary                 | 21923
footway                  | 21742
path                     | 21413
secondary                | 18444
...
\end{lstlisting}

\newpage

\section*{Question 5.a}

\begin{lstlisting}[language=SQL]
SELECT tags->'highway', SUM(ST_Length(linestring))
FROM ways
WHERE tags ? 'highway'
GROUP BY tags->'highway'
ORDER BY SUM(ST_Length(linestring)) DESC;
\end{lstlisting}

\begin{lstlisting}
?column?                 |         sum          
-------------------------+----------------------
unclassified             |     435.285712879273
track                    |     343.366696791793
residential              |     210.396982404606
tertiary                 |     208.294200085326
path                     |     134.367018999509
...
\end{lstlisting}

\section*{Question 5.b}

La documentation dit : \textit{Geometry: Measurements are in the units of the spatial reference system of the geometry.}

\begin{lstlisting}[language=SQL]
SELECT *
FROM spatial_ref_sys
WHERE srid = (
    SELECT ST_SRID(linestring) FROM ways LIMIT 1
);
\end{lstlisting}

Le syst�me de r�f�rence est WGS84 et l'unit� est le degr� (\textit{degree}).

\section*{Question 5.c}

On travaille avec des g�ographies, dont l'unit� est le m�tre.

\begin{lstlisting}[language=SQL]
SELECT tags->'highway', SUM(ST_Length(linestring::geography))/1000
FROM ways
WHERE tags ? 'highway'
GROUP BY tags->'highway'
ORDER BY SUM(ST_Length(linestring)) DESC;
\end{lstlisting}

\newpage

\section*{Question 6}

On r�cup�re tous les b�timents de l'Ensimag depuis la table \verb?relation_members?.

\begin{lstlisting}[language=SQL]
SELECT SUM(ST_Area(ST_MakePolygon(ways.linestring)::geography))
FROM relations
JOIN relation_members ON relations.id = relation_members.relation_id
JOIN ways ON relation_members.member_id = ways.id
WHERE
    relations.tags->'name' = 'Ensimag' AND
    relation_members.member_type = 'W';
\end{lstlisting}

\begin{lstlisting}
st_area      
------------------
635.243050458852
(1 row)
\end{lstlisting}

\newpage

\section*{Question 7}

\begin{lstlisting}[language=SQL]
SELECT quartier.quartier, count(ways.id)
FROM quartier, ways
WHERE
    ways.tags->'amenity' = 'school' AND
    ST_Contains(
        quartier.the_geom,
        ST_Transform(
            ST_MakePolygon(
                ST_AddPoint(
                    ways.linestring,
                    ST_StartPoint(ways.linestring)
                )
            ),
            2154
        )
    )
GROUP BY quartier.quartier
ORDER BY count(ways.id) DESC;
\end{lstlisting}

\begin{lstlisting}
quartier             | count 
---------------------+-------
BERRIAT ST BRUNO     |    13
CENTRE VILLLE        |    12
EXPOSITION-BAJATIERE |     9
RONDEAU-LIBERATION   |     7
EAUX CLAIRES         |     6
ABBAYE-JOUHAUX       |     5
...
\end{lstlisting}

\newpage

\section*{Question 8 (incomplet)}

On d�finit le centre g�ographique d'une r�gion comme la municipalit� ayant une
distance minimale avec sa municipalit� la plus �loign�e. On d�finit la distance
entre deux villes comme celle entre les centro�des des centro�des de leurs b�timents.

On commence par r�cup�rer les municipalit�s :

\begin{lstlisting}[language=SQL]
SELECT *
FROM relations
WHERE tags->'boundary' = 'administrative' AND tags->'admin_level' = '8';
\end{lstlisting}

Une relation contient plusieurs �l�ments g�om�triques, list�s dans la table
\verb?relation_members? :

\begin{lstlisting}[language=SQL]
SELECT *
FROM relations
INNER JOIN relation_members ON relations.id = relation_members.relation_id
WHERE tags->'boundary' = 'administrative' AND tags->'admin_level' = '8';
\end{lstlisting}

\verb?relation_members.member_type? fait r�f�rence � la table contenant les donn�es
g�ographiques :

\begin{lstlisting}[language=SQL]
SELECT DISTINCT(member_type)
FROM relation_members;
\end{lstlisting}

\begin{lstlisting}
member_type 
-------------
W
R
N
(3 rows)
\end{lstlisting}

Les b�timents sont stock�s dans la table \verb?ways?. On calcule le centro�d
de chaque municipalit� :

\begin{lstlisting}[language=SQL]
SELECT relations.id, ST_Centroid(ST_Union(ways.linestring))
FROM relations
JOIN relation_members ON relations.id = relation_members.relation_id
JOIN ways ON relation_members.member_id = ways.id
WHERE
    relations.tags->'boundary' = 'administrative' AND
    relations.tags->'admin_level' = '8' AND
    relation_members.member_type = 'W'
GROUP BY relations.id;
\end{lstlisting}

Puis, pour chaque ville, on calcule sa distance aux autres, prend celle maximale
et retourne la ville ayant la plus petite :

\begin{lstlisting}[language=SQL]
\end{lstlisting}

\section*{Question 9 (incomplet)}


\end{document}
